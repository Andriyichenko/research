%tcolorboxのテストと設定要覧(参考サイト: https://texmedicine.hatenadiary.jp/entry/2015/12/17/000339)
\documentclass[dvipdfmx]{jsarticle}
\usepackage{tikz}
\usetikzlibrary{positioning, intersections, calc, arrows.meta, math, through}
\usepackage{tcolorbox}
\usepackage{xcolor}
\definecolor{salmon}{rgb}{1.0, 0.55, 0.41}
\usepackage{enumerate}
\usepackage{lipsum}
\tcbuselibrary{theorems,breakable,skins,raster}
\tcbuselibrary{xparse} %preamble
\DeclareTColorBox{simplesquarebox}{ o m O{.5} O{} }% 
  {empty, left=2mm, right=2mm, top=-1mm, attach boxed title to top left={xshift=1.2zw},
  boxed title style={empty,left=-2mm,right=-2mm}, colframe=black, coltitle=black, coltext=black, breakable,  
  underlay unbroken={\draw[black,line width=#3pt]
    (title.east) -- (title.east-|frame.east) -- (frame.south east) -- (frame.south west) -- (title.west-|frame.west) -- (title.west); },
  underlay first={\draw[black,line width=#3pt](title.east) -- (title.east-|frame.east) -- (frame.south east) ;
    \draw[black,line width=#3pt] (frame.south west) -- (title.west-|frame.west) -- (title.west); },
  underlay middle={\draw[black,line width=#3pt](frame.north east) -- (frame.south east) ;
    \draw[black,line width=#3pt](frame.south west) -- (frame.north west) ;},
  underlay last={\draw[black,line width=#3pt](frame.north east) -- (frame.south east) -- (frame.south west) -- (frame.north west) ;},
  fonttitle=\gtfamily, IfValueTF={#1}{title=【#2】〈#1〉}{title=【#2】},#4}%watermark text
\begin{document}
\begin{tcolorbox}[enhanced,
  colframe=salmon,
  colback=salmon!20!white,
  coltitle=black,
  drop shadow, title=My box]
box contents
\end{tcolorbox}

\begin{simplesquarebox}[subtitle]{My box}[1.5][watermark text={Andre YI}]
  \lipsum[2]
  \end{simplesquarebox}

  \begin{tabular}{lccc}
    \hline
    状態変化 & $Q^{in}$ & $\varDelta U$ & $W^{out}$ \\
    \hline \hline
    定積変化 & $nC_V\varDelta T$ & $nC_V\varDelta T$ & 0 \\
    定圧変化 & $nC_P\varDelta T$ & $nC_V\varDelta T$ & $nR\varDelta T$ \\
    等温変化 & $nRT\log\frac{V_1}{V_0}$ & 0 & $nRT\log\frac{V_1}{V_0}$ \\
    断熱変化 & 0 & $nC_V\varDelta T$ & $-nC_V\varDelta T$ \\
    \hline
  \end{tabular}
\end{document}