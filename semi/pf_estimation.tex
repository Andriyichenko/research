\documentclass[dvipdfmx,a4paper,12pt]{jarticle}
\usepackage{amsmath,amssymb,amsfonts,amsthm}
\usepackage[dvipdfmx]{graphicx}
\usepackage{tikz}
\usetikzlibrary{positioning, intersections, calc, arrows.meta, math, through, shadows}
\usepackage{tcolorbox}
\tcbuselibrary{theorems,breakable}
\usepackage{enumerate}
\usepackage{mathtools}
\usepackage{otf}
\usepackage{xspace}
\usepackage{newpxtext}
\usepackage[utf8]{inputenc} %中国語コンパイル環境-cjkホットショット
\usepackage{CJKutf8,CJKspace,CJKpunct} %中国語コンパイル環境
\usepackage{pgfplots}
\pgfplotsset{compat=1.18}
\usepackage{okumacro} %漢字ruby
\renewcommand{\abstractname}{注意事項}
\newtagform{textbf}[	extbf]{[}{]}
\usetagform{textbf}
\newcommand*{\ie}{\textbf{\textit{i.e.}}\@\xspace}
\renewcommand{\qedsymbol}{$\blacksquare$}
\newtcbtheorem[]{reidai}{例題}
{fonttitle=\gtfamily\sffamily\bfseries\upshape\large,
colframe=black,colback=black!15!white,
rightrule=1pt,leftrule=1pt,bottomrule=2pt,
colbacktitle=black,theorem style=standard,breakable,arc=10pt}
{tha}
\renewcommand{\thefootnote}{\arabic{footnote}}
\newtheoremstyle{mystyle}%
  {}%                      % 上部スペース
  {}%                      % 下部スペース
  {}%                      % 本文フォント
  {}%                      % 1行目のインデント量
  {\bfseries}%             % 見出しフォント
  :%                       % 見出し後の句読点
  { }%                     % 見出し後のスペース
  {\thmname{#1}\thmnumber{ #2}\thmnote{ (#3)}}
\theoremstyle{mystyle}
% \setcounter{section}{0}
% \stepcounter{section}
% セクションカウンターを使用するが、表示はしない新しいセクションコマンドを作成
\newtheorem{dfn}{\texttt{Def.}}[section]
\newtheorem{exm}[dfn]{\texttt{Ex.}}
\newtheorem{prop}[dfn]{\texttt{Prop.}}
\newtheorem{lem}[dfn]{\texttt{Lem.}}
\newtheorem{thm}[dfn]{\texttt{Thm.}}
\newtheorem{cor}[dfn]{\texttt{Cor.}}
\newtheorem{rem}[dfn]{\texttt{Rem.}}
\newtheorem{fact}[dfn]{\texttt{Fact}}
\renewcommand{\qedsymbol}{$\blacksquare$}
\usepackage{lipsum} % 用于生成示例文本
\usepackage{float} % 强制浮动
\usepackage{tikz} % 用于定位
%排版
\newcommand{\kai}%解答
{\noindent
\begin{tikzpicture}[scale=0.2, baseline=2.8pt]
\draw (3.3,1.2) node{\large\textgt{解 答}};
\draw[thick, rounded corners=3pt,] (0,0)--(6.5,0)--(6.5,2.4)--(0,2.4)--cycle;
\end{tikzpicture}}
\newcommand{\shomei}%証明
{\noindent
\begin{tikzpicture}[scale=0.2, baseline=2.8pt]
\draw (3.3,1.2) node{\textgt{証 明}};
\draw[double,thick,rounded corners=3pt,] (0,0)--(6.5,0)--(6.5,2.4)--(0,2.4)--cycle;
\end{tikzpicture};}
%補足
\newcommand{\hosoku}{\noindent
\begin{tikzpicture}[scale=0.2, baseline=2.8pt]
\draw (6,1) node{\large\textgt{補足}};
\fill (0,1)--(1,0)--(2,1)--(1,2)--cycle;
\fill[gray] (1,1)--(2,0)--(3,1)--(2,2)--cycle;
\fill (2,1)--(3,0)--(4,1)--(3,2)--cycle;
\fill (10,1)--(11,0)--(12,1)--(11,2)--cycle;
\fill[gray] (9,1)--(10,0)--(11,1)--(10,2)--cycle;
\fill (8,1)--(9,0)--(10,1)--(9,2)--cycle;
\end{tikzpicture};}
%注意
\newcommand{\chui}{\noindent
\begin{tikzpicture}[scale=0.2, baseline=2.8pt]
\fill (0,0)--(6.5,0)--(6.5,2.2)--(0,2.2);
\draw (3.3,1) node[white]{\large\textgt{注意!}};
\draw[thick] (0,0)--(6.5,0)--(6.5,2.2)--(0,2.2)--cycle;
\end{tikzpicture};}
\title{\vspace{-3cm}\textbf{\Large 推定関数$\hat{p}$は確率密度関数の証明}}  %タイトル
\author{\texttt{YI Ran} - $\mathnormal{21122200512}$\\ \texttt{andreyi@outlook.jp}}  %著者名
\date{}  %日付
\begin{document}
\maketitle
%\vspace{-0.4cm}
%\begin{figure}[H]
%\centering
%\begin{tikzpicture}[remember picture, overlay]
%   \node[anchor=north east] at (current page.north east) {%
%        \includegraphics[width=2cm]{pics/qr.png} % 修正图片地址
%    };
%    \node[anchor=north east, yshift=-2cm] at (current page.north east) {デジタル版はここ};
%\end{tikzpicture}
%\label{fig:my_label}
%\end{figure}
%\begin{abstract} %概要
  %注意事項
%\end{abstract}
%\begin{reidai}{2次方程式}{解答}
%\end{reidai}
%\begin{proof}
%\end{proof}
\section*{\textbf{問題}}

\noindent
$W_{\sigma}(x,x_k)=\dfrac{1}{\sqrt{2\pi}\sigma} e^{-\frac{1}{2\sigma^2}|x-x_k|^2}  $とし、その推定関数を$\hat{p}(x)=\displaystyle\dfrac{1}{N}\sum_{k=1}^{N}W_{\sigma}(x,x_k)$とする。$\hat{p}(x)$は密度関数になることを証明せよ。

\section*{\textbf{解答}}

\begin{proof}
  非負化と正規化から証明します。\\

\noindent
\textbf{\large 非負化の証明:}\\

$W_{\sigma}(x,x_k)=\dfrac{1}{\sqrt{2\pi}\sigma} e^{-\frac{1}{2\sigma^2}|x-x_k|^2} \sim \mathcal{N}(x_k,\sigma^2)$なので、正規分布の密度関数の定義によって、 $ W_{\sigma}(x,x_k) \geq 0$ が成り立つことがわかる。
したがって、 $ \hat{p}(x)=\displaystyle\dfrac{1}{N}\sum_{k=1}^{N}W_{\sigma}(x,x_k) \geq 0 $ となる。\\

\noindent
\textbf{\large 正規化の証明:}\\

\noindent
$$
\int_{-\infty}^{\infty} \hat{p}(x) dx = \int_{-\infty}^{\infty} \left( \dfrac{1}{N}\sum_{k=1}^{N}W_{\sigma}(x,x_k) \right) dx \overset{N<\infty}{=} \dfrac{1}{N}\sum_{k=1}^{N} \int_{-\infty}^{\infty} W_{\sigma}(x,x_k) dx
$$

\noindent
ここで、 $ \displaystyle\int_{-\infty}^{\infty} W_{\sigma}(x,x_k) =\int_{-\infty}^{\infty} \dfrac{1}{\sqrt{2\pi}\sigma}e^{-\frac{1}{2\sigma^2} |x-x_k|^2}=1$であるので、
$ \displaystyle\int_{-\infty}^{\infty} \hat{p}(x) dx=\displaystyle\dfrac{1}{N}\sum_{k=1}^{N}\int_{-\infty}^{\infty}W_{\sigma}(x,x_k)=\dfrac{1}{N}\cdot N= 1$ となる。\ie $\displaystyle\int_{-\infty}^{\infty} \hat{p} (x) dx = 1$である。\\

\noindent
以上より、推定関数 $ \hat{p}(x) $は確率密度関数の性質を満たすことがわかる。
\end{proof}

\end{document}