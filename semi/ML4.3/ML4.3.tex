\documentclass[dvipdfmx,a4paper,11pt]{jsarticle}
\usepackage{ml4.3}

\begin{document}
\maketitle
\section{概要}
\thispagestyle{plain}
最適化問題における「関数の最小値を探索する」というプロセスは、物理学において\textbf{「谷(ポテンシャル関数の形状)の中に置かれたボールが、重力に従って底まで転がり落ちる運動」}として解釈することができます。
本節では、質量 $m=1$ のボールの運動を解析することで、最急降下法の物理的な意味と、単純な物理モデル(摩擦なし)では収束しない理由を数学的に解明する。
\vspace{1em}
\section{運動方程式の導出(ラグランジュ形式)}
\noindent
ボールの状態は、位置 $x$ と速度 $v$ のペア $s = (x, v)$ によって記述される。これを\textbf{相空間 (phase space)\footnotemark[1]} と呼ぶ。また、ボールには以下の2つのエネルギーが作用する。
\footnotetext[1]{相空間とは、全ての可能な状態の集合を指す物理学の用語。}
\begin{itemize}
    \item \textbf{運動エネルギー (Kinetic Energy):} $E_k = \frac{1}{2}\|v\|^2$ \quad (※質量 $m=1$のため )
    \item \textbf{位置エネルギー (Potential Energy):} $E_p = f(x)$ \quad (※目的関数の高さ)
\end{itemize}

\noindent
物理学における\textbf{ラグランジュ量 (Lagrangian)} $L$ は、運動エネルギーと位置エネルギーの\textbf{差}として定義されます。
\begin{equation}
    L(x, \dot{x}) = E_k - E_p = \frac{1}{2}\|\dot{x}\|^2 - f(x) \label{eq:lagrangian}
\end{equation}
ここで、速度 $v$ は位置の時間微分 $\dot{x}$ と等しいことに注意してください。

\begin{reidai}{Euler-Lagrangian 方程式による運動方程式の導出}{newton}
ラグランジュ量 $L$ を用いたEuler-Lagrangian 方程式:
\[ \frac{d}{dt} \left( \frac{\partial L}{\partial \dot{x}} \right) = \frac{\partial L}{\partial x} \]
を用いて、ボールの運動方程式 $\ddot{x}(t) = -\nabla f(x(t))$ を導出しなさい。
\end{reidai}
\vspace{0.5em}
\begin{proof}
 \ \\
\noindent
式[\ref{eq:lagrangian}]を各項について偏微分する\\
最初は\textbf{左辺の計算(運動量項)}$L$ を $\dot{x}$ で偏微分する。$f(x)$ は $\dot{x}$ に依存しない定数項として扱われる。
\[ \frac{\partial L}{\partial \dot{x}} = \frac{\partial}{\partial \dot{x}}\left( \frac{1}{2}\dot{x}^T \dot{x} \right) = \dot{x} = v \]
これをさらに時間 $t$ で微分すると
\[ \frac{d}{dt} v = \dot{v} = \footnotemark[2]\ddot{x}\]
\footnotetext[2]{$\dot{v} = \frac{d}{dt}v = \frac{d}{dt}\left(\frac{d}{dt}x\right) = \frac{d^2}{dt^2}x = \ddot{x} = -\nabla f(x)$. ここで、$\ddot{x}$は加速度であり、\textbf{エックス・ツー・ドット}と呼ぶ。}
\noindent
次に、\textbf{右辺の計算(力項)}$L$ を $x$ で偏微分する。$\|\dot{x}\|^2$ は $x$ に依存しないため消える。
\[ \frac{\partial L}{\partial x} = -\frac{\partial}{\partial x}f(x) = -\nabla f(x) \]
\noindent
これらを等号で結ぶと、以下のようになる。
\begin{equation}
 \ddot{x}(t) = -\nabla f(x(t)) \label{eq:ddotx}
\end{equation}
これは、質量 $m=1$、力 $F = -\nabla f(x)$ とした場合のニュートンの運動方程式 ($F=ma$) そのものである。
\end{proof}
\section{エネルギー保存則とハミルトニアン}

次に、運動中の「総エネルギー」の変化を考える。
総エネルギー $E_{tot}$ は、運動エネルギーと位置エネルギーの「和」である。
\begin{equation}
    E_{tot}(t) = E_k(t) + E_p(t) = \frac{1}{2}\|\dot{x}(t)\|^2 + f(x(t))
\end{equation}

\begin{reidai}{エネルギー保存則の証明}{energy_conservation}
運動方程式 $\ddot{x} = -\nabla f(x)$ に従うとき、総エネルギー $E_{tot}(t)$ が時間変化しないこと、\ie $\frac{d}{dt}E_{tot}(t) = 0$ を証明しなさい。
\end{reidai}
\vspace{0.5em}
\begin{proof}
  \ \\
\noindent
総エネルギー式を時間 $t$ で微分すると、以下のようになる。
\begin{align*}
    \frac{d}{dt} E_{tot}(t) &= \frac{d}{dt} \left( \frac{1}{2}\dot{x}(t)^T \dot{x}(t) + f(x(t)) \right) \\
    &= \dfrac{1}{2}((\dfrac{d}{dt}\dot{x}(t))^T \dot{x}(t) + \dot{x}(t)^T \dfrac{d}{dt}\dot{x}(t)) + \nabla f(x(t))^T \dfrac{d}{dt}(x(t)) \\
    &= \dot{x}(t)^T \ddot{x}(t) + \dot{x}(t)^T \nabla f(x(t))  \\
    &= \dot{x}(t)^T \left( \ddot{x}(t) + \nabla f(x(t)) \right)
\end{align*}
ここで、前節の[\ref{eq:ddotx}]で導いた運動方程式より $\ddot{x}(t) + \nabla f(x(t)) = 0$ が成り立つ。
したがって、
\[ \frac{d}{dt} E_{tot}(t) = \dot{x}(t)^T \cdot 0 = 0 \]
となり、エネルギーは保存されることが証明された。
\end{proof}
\hosoku\\
\noindent
この総エネルギーを相空間 $(x, v)$ 上の関数として定義したものを\textbf{ハミルトニアン (Hamiltonian)} $H$ と呼ぶ。
\[ H(x, v) = \frac{1}{2}\|v\|^2 + f(x) \]
エネルギーが保存されるということは、摩擦のないワンの中のボールは、永遠に転がり続け、決して底(最小値)に静止しないことを示唆している。

\section{なぜ「摩擦なし」では収束しないのか}

運動方程式は以下の1階連立常微分方程式(ODE)系として書き直せる。
\begin{empheq}[left=\empheqlbrace]{align}
    \dot{x}(t) &= v(t) = \frac{\partial H}{\partial v} \label{eq:ode1} \\
    \dot{v}(t) &= -\nabla f(x(t)) = -\frac{\partial H}{\partial x} \label{eq:ode2}
\end{empheq}
この系におけるベクトル場の発散 (Divergence) を計算することで、収束性がわかる。

\begin{reidai}{ベクトル場の発散の計算}{divergence}
相空間上のベクトル場 $X = (\dot{x}, \dot{v}) = \left( \frac{\partial H}{\partial v}, -\frac{\partial H}{\partial x} \right)$ の発散 $\textbf{div}X\ $(\ie $\nabla \cdot X$) を計算しなさい。
\end{reidai}
\vspace{0.5em}
\kai\\
\noindent
発散の定義に従い、計算すると。
\begin{align*}
    \text{div}X &= \frac{\partial}{\partial x}(\dot{x}) + \frac{\partial}{\partial v}(\dot{v}) \\
    &= \frac{\partial}{\partial x}\left( \frac{\partial H}{\partial v} \right) + \frac{\partial}{\partial v}\left( -\frac{\partial H}{\partial x} \right) \\
    &= \frac{\partial^2 H}{\partial x \partial v} - \frac{\partial^2 H}{\partial v \partial x}
\end{align*}
滑らかな関数 $H$ において偏微分の順序交換が可能 ($\frac{\partial^2 H}{\partial x \partial v} = \frac{\partial^2 H}{\partial v \partial x}$) であるため、
\[ \text{div}X = 0 \]
となる。
\newpage
\hosoku\\
\noindent
$\text{div}X = 0$ は、この物理系の流れが\textbf{非圧縮 (incompressible)}であることを意味する。
これは、相空間内の状態の集合(体積)が、時間発展しても\textbf{縮小しない}ことを意味する。
最小点(平衡点)への収束は、状態空間の体積が一点に収縮することと等しいであるが、非圧縮性によりこれが不可能である。
したがって、単純な物理モデル(摩擦なし)では、最適解へ収束せず、その周りを永遠に振動し続ける。収束させるには\textbf{摩擦項}の導入が必要不可欠である。

\newpage
\appendix
\section{ハミルトニアン(Hamiltonian)に対する最急降下法}
\begin{reidai}{補足}{add}
  もし、ハミルトニアン $H(x, v)$ 自体を目的関数と見なし、最急降下法を適用するとどうなるかを確認する。
  ここで、平衡状態$S^* = (x^*, v^*)$,初期状態$S^0 = (x^0, v^0)$とする。
\end{reidai}
\vspace{0.5em}
$H$の勾配は以下のようになる
\[ \nabla_s H(s) = \left( \frac{\partial H}{\partial x}, \frac{\partial H}{\partial v} \right) = (\nabla f(x), v) \]

これを用いた更新式 $s_{n+1} = s_n - \delta \nabla_s H(s_n)$ は、成分ごとに書くと以下の通りである。
\begin{align*}
    x^{n+1} &= x^n - \delta \nabla f(x^n) \\
    v^{n+1} &= v^n - \delta v^n = (1 - \delta)v^n
\end{align*}

ここで、$v^n$とはn回目の反復のインデックスである。上の結果を解析すると
\begin{enumerate}
    \item $v^n$ の更新式は $v^{n+1} = (1-\delta)v^n\implies v^n = (1-\delta)^n v^0$ となり、$\delta$ が小さければ$v^n \to v^* = 0$ ($n\to \infty$)\ \ie 速度は消える。
    \item $x$ の更新式は $x^{n+1} = x^n - \delta \nabla f(x^n)$ となり、これは通常の最急降下法と全く同じである。
\end{enumerate}
\vspace{1em}
つまり、エネルギー関数に対して単純に最急降下法を行っても、\textbf{「速度成分がなくなり、単
なる(運動量のない)最急降下法に戻ってしまう」}という結果になる。運動量(慣性)の効
果を得るには、運動方程式自体に摩擦項を組み込むアプローチが必要である。\ie $\mathbf{div} < 0$が必要である。

\end{document}